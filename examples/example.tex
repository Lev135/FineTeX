% That was a block of definitions. It should be the first environment in file
% By this moment there is no way to import definitions from other file
% so we have to declare all base environments and commands in every file separately

% Note, that there are no comments in .ttex format by this moment. 
% These comments are just a paragraph, that will be translated in .tex file 
% (and then removed during the compilation of .pdf file)
% As a consequence we can't use comments in @Define environment and 
% in other places, except those in which paragraphs are allowed.
% In addition, you cannot use back quotes in them, since they will be parsed
% as an inline math expression.

% By now we should add \documentclass and \usepackage commands
% also as a plain text
\documentclass[12pt,a4paper,oneside]{article}
\usepackage[utf8]{inputenc}
\usepackage{amsmath}
\usepackage{amssymb}

\begin{document}
% Everything here (2 spaces-indented) will be in \begin{document} .. \end{document}
% Indentation must be exactly 2 spaces

This is a simple paragraph
Which continues at the next line

And this is the next paragraph

\section{lists} 
% By this moment there is no way to take arguments in environment,
% so we use a paragraph with TeX command here to start a section
@Prefs style is very useful for items. 
Note that prefs should be indented more then surrounding paragraphs
\begin{itemize}
\item enumerate (this text is in paragraph inside an item):
\begin{enumerate}
\item one (it should be more indented then paragraph)
\item two
\item three
\end{enumerate}
\item itemize:
\begin{itemize}
\item one
\item two
\item three
\end{itemize}
\end{itemize}

Consecutive elements of the list are framed by the itemize/enumerate environment,
if there is no empty lines between them. So:
\begin{enumerate}
\item one
\item two
\item three
\end{enumerate}

\begin{enumerate}
\item one again
\item two
\end{enumerate}

\section{Math}
Inline math can be inserted in paragraph in back quotes: $a + b = c$ 
--- this is inline math

Fore block math @Prefs syntax is used: 
\begin{align*}
a + b = c 
\end{align*}
Next lines will be framed by align* environment and \verb.\\. 
will be added between them automatically
\begin{align*}
a + b     &= c
\\(a + b)^2 &= d
\end{align*}

\begin{align*}
a + b + c &= 0
\end{align*}
Note that the last equation will framed by separate align*, because
there is an empty line before it. 

To be continued... 
\end{document}
