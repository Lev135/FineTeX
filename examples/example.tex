\documentclass[12pt,a4paper,oneside]{article}
\usepackage[utf8]{inputenc}
\usepackage{amsmath}
\usepackage{amssymb}
% This comment will be in output


\begin{document}
  \section{Basics}
    Basic elements of these language are Paragraphs, Environments and Prefs

    \subsection{Paragraphs}
      Paragraphs here, as well as in TeX are separated by empty lines:

      This is a simple paragraph Which continues at the next line

      And this is the next paragraph

    \subsection{Environments}
      

    \subsection{Prefs}
      

  \section{Lists}
    \texttt{@Prefs} style is very useful for items. Note that prefs should be indented more then
    surrounding paragraphs
    \begin{itemize}
      \item enumerate (this text is in paragraph inside an item so it can be continued on the next
            line):
            \begin{enumerate}
              \item one (it should be more indented then paragraph)
              \item two
              \item three
            \end{enumerate}
      \item itemize:
            \begin{itemize}
              \item one
              \item two
              \item three
            \end{itemize}
    \end{itemize}

    Consecutive elements of the list are framed by the itemize/enumerate environment, iff there are
    no empty lines between them. So:
    \begin{enumerate}
      \item one
      \item two
      \item three
    \end{enumerate}

    \begin{enumerate}
      \item one again
      \item two
    \end{enumerate}

  \section{Math}
    Inline math can be inserted in paragraph in back quotes: $a + b = c$  --- this is inline math

    For block math \texttt{@Prefs} syntax is used:
    \begin{align*}
      a + b = c
    \end{align*}
    Next lines will be framed by align* environment and \verb.\\. will be added between them
    automatically
    \begin{align*}
      a + b &= c\\
      (a + b)^2 &= d
    \end{align*}

    \begin{align*}
      a + b + c &= 0
    \end{align*}
    Note that the last equation will framed by separate align*, because there is an empty line
    before it.

  To be continued...
  \section{Known bugs and strange features}
    There is nothing here at this moment
\end{document}
