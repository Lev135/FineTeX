\documentclass[12pt,a4paper,oneside]{article}
\usepackage[left=2cm, right=2cm, top=1.3cm, bottom=2cm]{geometry}
\usepackage[utf8]{inputenc}
\usepackage[russian]{babel}
\usepackage{amsthm}
\usepackage{amsmath}
\usepackage{amssymb}

\newtheorem{definition}{Определение}
\newtheorem{theorem}{Теорема}
\newtheorem{proposition}{Предложение}
\newtheorem{lemma}{Лемма}
\newtheorem*{corollary}{Следствие}

\begin{document}
  Обозначения модальностей:
  \begin{itemize}
    \item $\Diamond_0$ и $\Diamond_1$ -- модальности $GLB$, $J$
    \item $\Diamond_* A := \Diamond_0(\Diamond_1 T \wedge A)$ -- доказуемость в предположении
          1-непротиворечивости
    \item $\Diamond A := \Diamond_1 A \vee \Diamond_* A$ -- доказуемость арифметики Нибергаля
          относительно PA
  \end{itemize}

  \begin{lemma}
    $J \vdash K(\Diamond), GL(\Diamond_*), GL(\Diamond_*^2), GL(\Diamond_0\Diamond_1\cdot),
    GL(\Diamond_1 T \wedge \Diamond_*\cdot)$\\ Под этой записью мы подразумеваем, что в $J$
    доказываются аксиомы соответствующих теорий и усиление по соответствующей модальности является
    допустимым правилом вывода в $J$.
  \end{lemma}
  \begin{proof}
    Докажем выводимость аксиом Лёба.
    \begin{itemize}
      \item $J \vdash \Diamond_* p \rightarrow \Diamond_*(p \wedge \neg\Diamond_* p)$\\ Применяя
            аксиому Лёба для $\Diamond_0$:
            \begin{align*}
              J \vdash \Diamond_0(\Diamond_1 T \wedge p) \rightarrow \Diamond_0(\Diamond_1 T \wedge
              p \wedge \neg\Diamond_0(\Diamond_1 T \wedge p))
            \end{align*}
            с точностью до обозначения $\Diamond_*\cdot = \Diamond_0(\Diamond_1 T \wedge \cdot)$ это
            и есть требуемое
      \item $J \vdash \Diamond_*^2 p \rightarrow \Diamond_*^2(p \wedge \neg\Diamond_*^2 p)$\\ Из
            предыдущего пункта, пользуясь нормальностью $\Diamond_*$:
            \begin{align*}
              J \vdash \Diamond_*^2 p \rightarrow \Diamond_*^2(p \wedge \neg\Diamond_* p)
            \end{align*}
            Осталось воспользоваться тем, что
            $GL(\Diamond_*) \vdash \Diamond_*^2 p \rightarrow \Diamond_* p$
      \item $J \vdash \Diamond_0\Diamond_1 p \rightarrow \Diamond_0\Diamond_1(p \wedge
            \neg\Diamond_0\Diamond_1 p)$\\ По аксиоме Лёба
            \begin{align*}
              J \vdash \Diamond_0\Diamond_1 p &\rightarrow \Diamond_0(\Diamond_1 p \wedge
              \neg\Diamond_0\Diamond_1 p)\\
              \Diamond_1 p &\rightarrow \Diamond_1(p \wedge \neg\Diamond_1 p)
            \end{align*}
            Пользуясь тем, что
            $J \vdash \neg\Diamond_0\Diamond_1 p \rightarrow \Box_1(\neg\Diamond_0\Diamond_1 p)$ по
            нормальности получаем требуемое
      \item $J \vdash \Diamond_1 T \wedge \Diamond_* p \rightarrow \Diamond_1 T \wedge \Diamond_*(p
            \wedge (\neg\Diamond_1 T \vee \neg\Diamond_* p))$\\ Следует из аксиомы Лёба для
            $\Diamond_*$
    \end{itemize}
  \end{proof}

  \begin{lemma}
    \begin{align*}
      J \vdash &\Diamond\Diamond_* p \rightarrow \Diamond_* p,\\
      &\Diamond_*\Diamond_1 p \leftrightarrow \Diamond_0\Diamond_1 p,\\
      &\Diamond_1\Diamond_0 p \leftrightarrow \Diamond_1 T \wedge \Diamond_0 p,\\
      &\Diamond_1\Diamond_* p \leftrightarrow \Diamond_1 T \wedge \Diamond_* p
    \end{align*}
  \end{lemma}

  \begin{proposition}
    $J \vdash \Diamond(p \wedge \Diamond(q \wedge \Diamond r)) \rightarrow \Diamond q \vee
    \Diamond(p \wedge \Diamond r) \wedge \Diamond r$
  \end{proposition}
  \begin{proof}
    Раскроем $\Diamond$ в посылке
    \begin{align*}
      \Diamond(p \wedge \Diamond_*(q \wedge \Diamond r)) \rightarrow \Diamond\Diamond_* q,
      \rightarrow \Diamond_* q, \rightarrow \Diamond q
    \end{align*}

    \begin{align*}
      \Diamond(p \wedge \Diamond_1(q \wedge \Diamond_1 r)) &\rightarrow \Diamond\Diamond_1(q \wedge
      \Diamond_1 r), \equiv \Diamond_1^2(q \wedge \Diamond_1 r) \vee \Diamond_*\Diamond_1(q \wedge
      \Diamond_1 r)\\
      \Diamond_1^2(q \wedge \Diamond_1 r) &\rightarrow \Diamond_1^2 q, \rightarrow \Diamond_1 q
      \rightarrow \Diamond q\\
      \Diamond_*\Diamond_1(q \wedge \Diamond_1 r) &\leftrightarrow \Diamond_0\Diamond_1(q \wedge
      \Diamond_1 r), \rightarrow \Diamond_0(q \wedge \Diamond_1 r), \rightarrow \Diamond_* q,
      \rightarrow \Diamond q
    \end{align*}

    \begin{align*}
      \Diamond(p \wedge \Diamond_1(q \wedge \Diamond_* r)) &\rightarrow
      \Diamond\Diamond_1\Diamond_* r, \rightarrow \Diamond_* r, \rightarrow \Diamond r\\
      \Diamond(p \wedge \Diamond_1(q \wedge \Diamond_* r)) &\rightarrow \Diamond(p \wedge
      \Diamond_1\Diamond_* r), \Diamond(p \wedge \Diamond r)
    \end{align*}
  \end{proof}
  \begin{corollary}
    $J \vdash \Diamond^3p \rightarrow \Diamond^2p$
  \end{corollary}

  \begin{lemma}
    Пусть $\Diamond_a$ и $\Diamond_b$ --- модальности логики $T$,
    $\Diamond_\star A \equiv \Diamond_a A \vee \Diamond_b A$.
    \begin{align*}
      T \vdash GL(\Diamond_a), GL(\Diamond_b), \Diamond_b\Diamond_a p \rightarrow \Diamond_a p,
      \Diamond_a\Diamond_b p \rightarrow \Diamond_\star p
    \end{align*}
    Тогда $T \vdash GL(\Diamond_\star)$
  \end{lemma}
  \begin{proof}
    Нормальность очевидна. \\ Из аксиомы Лёба для $\Diamond_a$
    \begin{align*}
      \Diamond_a p \rightarrow \Diamond_a(p \wedge \neg\Diamond_\star p) \vee \Diamond_a(p \wedge
      \neg\Diamond_a p \wedge \Diamond_b p)
    \end{align*}
    Из аксиомы Лёба для $\Diamond_b$
    \begin{align*}
      \Diamond_a(p \wedge \neg\Diamond_a p \wedge \Diamond_b p) \rightarrow \Diamond_a(p \wedge
      \neg\Diamond_a p \wedge \Diamond_b(p \wedge \neg\Diamond_b p))
    \end{align*}
    Откуда по нормальности, так как $\neg\Diamond_a p \rightarrow \Box_b\neg\Diamond_a p$
    \begin{align*}
      \Diamond_a(p \wedge \neg\Diamond_a p \wedge \Diamond_b p) &\rightarrow \Diamond_a(p \wedge
      \Diamond_b(p \wedge \neg\Diamond_\star p))\\
      &\rightarrow \Diamond_a\Diamond_b(p \wedge \neg\Diamond_\star p),\\
      &\rightarrow \Diamond_\star(p \wedge \neg\Diamond_\star p)
    \end{align*}
    Итак, $\Diamond_a p \rightarrow \Diamond_\star(p \wedge \neg\Diamond p)$. Аналогично получаем
    \begin{align*}
      \neg\Diamond_a p \wedge \Diamond_b p &\rightarrow \Box_b\neg\Diamond_a p \wedge \Diamond_b(p
      \wedge \neg\Diamond_b p)\\
      &\rightarrow \Diamond_b(p \wedge \neg\Diamond_\star p)
    \end{align*}
  \end{proof}

  \begin{proposition}
    $J \vdash GL(\Diamond^2\cdot)$
  \end{proposition}
  \begin{proof}
    \begin{align*}
      J \vdash \Diamond^2 p &\leftrightarrow \Diamond_1\Diamond_1 p \vee \Diamond_1\Diamond_* p \vee
      \Diamond_*\Diamond_1 p \vee \Diamond_*\Diamond_* p\\
      &\leftrightarrow \Diamond_1 T \wedge \Diamond_* p \vee \Diamond_0\Diamond_1 p \vee
      \Diamond_*^2 p
    \end{align*}
    Теперь дважды воспользуемся леммой. \\ Сначала для $\Diamond_a \equiv \Diamond_0\Diamond_1$ и
    $\Diamond_b \equiv \Diamond_*^2$.
    \begin{itemize}
      \item $J \vdash \Diamond_*^2\Diamond_0\Diamond_1 p \rightarrow \Diamond_0^3\Diamond_1 p,
            \rightarrow \Diamond_0\Diamond_1 p$
      \item $J \vdash \Diamond_0\Diamond_1\Diamond_*^2 p \rightarrow \Diamond_* p$
    \end{itemize}
    Осталось применить ту же лемму для $\Diamond_a \equiv \Diamond_1 T \wedge \Diamond_*\cdot$ и
    $\Diamond_b \equiv \Diamond_0\Diamond_1\cdot \vee \Diamond_*^2\cdot$.
    \begin{itemize}
      \item $J \vdash \Diamond_1 T \wedge \Diamond_*(\Diamond_0\Diamond_1 p \vee \Diamond_*^2 p)
            \rightarrow \Diamond_0\Diamond_0\Diamond_1 p \vee \Diamond_0\Diamond_*^2 p, \rightarrow
            \Diamond_0\Diamond_1 p \vee \Diamond_*^2 p$
      \item $J \vdash \Diamond_0\Diamond_1(\Diamond_1 T \wedge \Diamond_* p) \vee \Diamond_*^2
            (\Diamond_1 T \wedge \Diamond_* p) \rightarrow \Diamond_0\Diamond_1\Diamond_* p \vee
            \Diamond_*^3 p, \rightarrow \Diamond_0\Diamond_1 p \vee \Diamond_*^2 p$
    \end{itemize}
  \end{proof}

  \begin{corollary}
    $J \vdash GL(\Diamond\cdot \vee \Diamond^2\cdot)$, а точнее:
    \begin{itemize}
      \item $J \vdash \Diamond^2 p \rightarrow \Diamond^2(p \wedge \neg\Diamond p \wedge
            \neg\Diamond^2 p)$
      \item $J \vdash \Diamond p \rightarrow \Diamond(p \wedge \neg\Diamond p \wedge \neg\Diamond^2
            p) \vee \Diamond^2 p$
    \end{itemize}
  \end{corollary}
  \begin{proof}
    То, что $J \vdash GL(\Diamond\cdot \vee \Diamond^2\cdot)$ является следствием следующих пунктов
    \begin{itemize}
      \item Используя предыдущее предложение и нормальность
            \begin{align*}
              J \vdash \Diamond^2 p \rightarrow \Diamond^2(p \wedge \neg\Diamond p \wedge
              \neg\Diamond^2 p) \vee \Diamond^2(p \wedge \Diamond p \wedge \neg\Diamond^2 p)
            \end{align*}
            Так как $J \vdash \Diamond^3 p \rightarrow \Diamond^2 p$ и $K(\Diamond) \vdash
            \Diamond p \wedge \neg\Diamond^2 p \wedge \neg\Diamond^3 p \rightarrow \Diamond(p \wedge
            \neg\Diamond p \wedge \neg\Diamond^2 p)$
            \begin{align*}
              J \vdash \Diamond^2(p \wedge \Diamond p \wedge \neg\Diamond^2 p) \rightarrow
              \Diamond^2\Diamond(p \wedge \neg\Diamond p \wedge \neg\Diamond^2 p), \rightarrow
              \Diamond^2(p \wedge \neg\Diamond p \wedge \neg\Diamond^2 p)
            \end{align*}
      \item Заметим, что
            \begin{align*}
              K(\Diamond) \vdash \Diamond p \rightarrow \Diamond(p \wedge \neg\Diamond p \wedge
              \neg\Diamond^2 p) \vee \Diamond^2 p \vee \Diamond^3 p
            \end{align*}
            Откуда пользуясь тем, что $J \vdash \Diamond^3 p \rightarrow \Diamond^2 p$ получаем
            требуемое
    \end{itemize}
  \end{proof}
\end{document}
