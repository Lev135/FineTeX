\documentclass[12pt,a4paper,oneside]{article}
\usepackage[left=2cm, right=2cm, top=1.3cm, bottom=2cm]{geometry}
\usepackage[utf8]{inputenc}
\usepackage[russian]{babel}
\usepackage{amsthm}
\usepackage{amsmath}
\usepackage{amssymb}

\newtheorem{definition}{Определение}
\newtheorem{theorem}{Теорема}
\newtheorem{proposition}{Предложение}
\newtheorem{lemma}{Лемма}
\newtheorem*{corollary}{Следствие}

\begin{document}
  Обозначения модальностей:
  \begin{itemize}
    \item $\Diamond_0 $ и $\Diamond_1 $ -- модальности $GLB$, $J$
    \item $\Diamond_* A := \Diamond_0 (\Diamond_1 \top \wedge A)$ -- доказуемость в
          предположении 1-непротиворечивости
    \item $\Diamond A := \Diamond_1 A \vee \Diamond_* A$ -- доказуемость
          арифметики Нибергаля относительно PA
  \end{itemize}

  \begin{lemma}
    $J \vdash K(<>), GL(<*>), GL(<*>^2), GL(<0><1>\cdot), GL(<1>T \wedge <*>\cdot)$\\ Под
    этой записью мы подразумеваем, что в $J$ доказываются
    аксиомы соответствующих теорий и усиление по
    соответствующей модальности является допустимым
    правилом вывода в $J$.
  \end{lemma}
  \begin{proof}
    Докажем выводимость аксиом Лёба.
    \begin{itemize}
      \item $J \vdash \Diamond_* p \rightarrow \Diamond_* (p \wedge \neg \Diamond_* p)$\\
            Применяя аксиому Лёба для $\Diamond_0 $:
            \begin{align*}
              J \vdash \Diamond_0 (\Diamond_1 \top \wedge p) \rightarrow \Diamond_0 (\Diamond_1 \top
              \wedge p \wedge \neg \Diamond_0 (\Diamond_1 \top \wedge p))
            \end{align*}
            с точностью до обозначения $<*>\cdot = \Diamond_0 (\Diamond_1
            \top \wedge \cdot)$ это и есть требуемое
      \item $J \vdash \Diamond_* ^2 p \rightarrow \Diamond_* ^2(p \wedge \neg \Diamond_* ^2 p)$\\
            Из предыдущего пункта, пользуясь нормальностью
            $\Diamond_* $:
            \begin{align*}
              J \vdash \Diamond_* ^2 p \rightarrow \Diamond_* ^2(p \wedge \neg \Diamond_* p)
            \end{align*}
            Осталось воспользоваться тем, что $GL(<*>) \vdash
            \Diamond_* ^2 p \rightarrow \Diamond_* p$
      \item $J \vdash \Diamond_0 \Diamond_1 p \rightarrow \Diamond_0 \Diamond_1 (p \wedge \neg
            \Diamond_0 \Diamond_1 p)$\\ По аксиоме Лёба
            \begin{align*}
              J \vdash \Diamond_0 \Diamond_1 p &-> \Diamond_0 (\Diamond_1 p \wedge \neg \Diamond_0
              \Diamond_1 p)\\
              \Diamond_1 p &-> \Diamond_1 (p \wedge \neg \Diamond_1 p)
            \end{align*}
            Пользуясь тем, что $J \vdash \neg \Diamond_0 \Diamond_1 p \rightarrow
            \Box_1 (\neg \Diamond_0 \Diamond_1 p)$ по нормальности получаем
            требуемое
      \item $J \vdash \Diamond_1 \top \wedge \Diamond_* p \rightarrow \Diamond_1 \top \wedge
            \Diamond_* (p \wedge (\neg \Diamond_1 \top \vee \neg \Diamond_* p))$\\ Следует
            из аксиомы Лёба для $\Diamond_* $
    \end{itemize}
  \end{proof}

  \begin{lemma}
    > J |- &<><*>p -> <*>p, > &<*><1>p <-> <0><1>p, > &<1><0>p <-> <1>T /\ <0>p, > &<1><*>p <-> <1>T
    /\ <*>p
  \end{lemma}

  \begin{proposition}
    $J \vdash \Diamond (p \wedge \Diamond (q \wedge \Diamond r)) \rightarrow \Diamond q \vee
    \Diamond (p \wedge \Diamond r) \wedge \Diamond r$
  \end{proposition}
  \begin{proof}
    Раскроем $\Diamond $ в посылке
    \begin{align*}
      \Diamond (p \wedge \Diamond_* (q \wedge \Diamond r)) \rightarrow <><*>q, \rightarrow <*>q,
      \rightarrow \Diamond q
    \end{align*}

    \begin{align*}
      \Diamond (p \wedge \Diamond_1 (q \wedge \Diamond_1 r)) &-> \Diamond \Diamond_1 (q \wedge
      <1>r), \equiv \Diamond_1 ^2(q \wedge \Diamond_1 r) \vee \Diamond_* \Diamond_1 (q \wedge
      \Diamond_1 r)\\
      \Diamond_1 ^2(q \wedge \Diamond_1 r) &-> \Diamond_1 ^2 q, \rightarrow \Diamond_1 q \rightarrow
      \Diamond q\\
      \Diamond_* \Diamond_1 (q \wedge \Diamond_1 r) &<-> \Diamond_0 \Diamond_1 (q \wedge <1>r),
      \rightarrow \Diamond_0 (q \wedge <1>r), \rightarrow <*>q, \rightarrow \Diamond q
    \end{align*}

    \begin{align*}
      \Diamond (p \wedge \Diamond_1 (q \wedge \Diamond_* r)) &-> <><1><*>r, \rightarrow <*>r,
      \rightarrow \Diamond r\\
      \Diamond (p \wedge \Diamond_1 (q \wedge \Diamond_* r)) &-> \Diamond (p \wedge <1><*>r),
      \Diamond (p \wedge \Diamond r)
    \end{align*}
  \end{proof}
  \begin{corollary}
    $J \vdash \Diamond ^3p \rightarrow \Diamond ^2p$
  \end{corollary}

  \begin{lemma}
    Пусть $\Diamond_a $ и $\Diamond_b $ --- модальности логики $\top $,
    $\Diamond_\star A \equiv \Diamond_a A \vee \Diamond_b A$.
    \begin{align*}
      \top \vdash GL(<a>), GL(<b>), \Diamond_b \Diamond_a p \rightarrow <a>p, \Diamond_a \Diamond_b
      p \rightarrow \Diamond_\star p
    \end{align*}
    Тогда $\top \vdash GL(<x>)$
  \end{lemma}
  \begin{proof}
    Нормальность очевидна. \\ Из аксиомы Лёба для $\Diamond_a $
    \begin{align*}
      \Diamond_a p \rightarrow \Diamond_a (p \wedge \neg \Diamond_\star p) \vee \Diamond_a (p \wedge
      \neg \Diamond_a p \wedge \Diamond_b p)
    \end{align*}
    Из аксиомы Лёба для $\Diamond_b $
    \begin{align*}
      \Diamond_a (p \wedge \neg \Diamond_a p \wedge \Diamond_b p) \rightarrow \Diamond_a (p \wedge
      \neg \Diamond_a p \wedge \Diamond_b (p \wedge \neg \Diamond_b p))
    \end{align*}
    Откуда по нормальности, так как $\neg \Diamond_a p \rightarrow \Box_b
    \neg \Diamond_a p$
    \begin{align*}
      \Diamond_a (p \wedge \neg \Diamond_a p \wedge \Diamond_b p) &-> \Diamond_a (p \wedge
      \Diamond_b (p \wedge \neg \Diamond_\star p))\\
      &-> \Diamond_a \Diamond_b (p \wedge ~<x>p),\\
      &-> \Diamond_\star (p \wedge \neg \Diamond_\star p)
    \end{align*}
    Итак, $\Diamond_a p \rightarrow \Diamond_\star (p \wedge \neg \Diamond p)$.
    Аналогично получаем
    \begin{align*}
      \neg \Diamond_a p \wedge \Diamond_b p &-> \Box_b \neg \Diamond_a p \wedge \Diamond_b (p \wedge
      \neg \Diamond_b p)\\
      &-> \Diamond_b (p \wedge \neg \Diamond_\star p)
    \end{align*}
  \end{proof}

  \begin{proposition}
    $J \vdash GL(<>^2\cdot)$
  \end{proposition}
  \begin{proof}
    \begin{align*}
      J \vdash \Diamond ^2 p &<-> \Diamond_1 \Diamond_1 p \vee \Diamond_1 \Diamond_* p \vee
      \Diamond_* \Diamond_1 p \vee \Diamond_* \Diamond_* p\\
      &<-> \Diamond_1 \top \wedge \Diamond_* p \vee \Diamond_0 \Diamond_1 p \vee \Diamond_* ^2 p
    \end{align*}
    Теперь дважды воспользуемся леммой. \\ Сначала для
    $\Diamond_a \equiv \Diamond_0 \Diamond_1 $ и $\Diamond_b \equiv \Diamond_* ^2$.
    \begin{itemize}
      \item $J \vdash \Diamond_* ^2\Diamond_0 \Diamond_1 p \rightarrow <0>^3<1>p, \rightarrow
            \Diamond_0 \Diamond_1 p$
      \item $J \vdash \Diamond_0 \Diamond_1 \Diamond_* ^2 p \rightarrow \Diamond_* p$
    \end{itemize}
    Осталось применить ту же лемму для $\Diamond_a \equiv \Diamond_1
    \top \wedge <*>\cdot$ и $\Diamond_b \equiv <0><1>\cdot \vee <*>^2\cdot$.
    \begin{itemize}
      \item $J \vdash \Diamond_1 \top \wedge \Diamond_* (\Diamond_0 \Diamond_1 p \vee \Diamond_* ^2
            p) \rightarrow \Diamond_0 \Diamond_0 \Diamond_1 p \vee \Diamond_0 \Diamond_* ^2 p,
            \rightarrow \Diamond_0 \Diamond_1 p \vee \Diamond_* ^2 p$
      \item $J \vdash \Diamond_0 \Diamond_1 (\Diamond_1 \top \wedge \Diamond_* p) \vee \Diamond_* ^2
            (\Diamond_1 \top \wedge \Diamond_* p) \rightarrow \Diamond_0 \Diamond_1 \Diamond_* p
            \vee \Diamond_* ^3 p, \rightarrow \Diamond_0 \Diamond_1 p \vee \Diamond_* ^2 p$
    \end{itemize}
  \end{proof}

  \begin{corollary}
    $J \vdash GL(<>\cdot \vee <>^2\cdot)$, а точнее:
    \begin{itemize}
      \item $J \vdash \Diamond ^2 p \rightarrow \Diamond ^2(p \wedge \neg \Diamond p \wedge \neg
            \Diamond ^2 p)$
      \item $J \vdash \Diamond p \rightarrow \Diamond (p \wedge \neg \Diamond p \wedge \neg \Diamond
            ^2 p) \vee \Diamond ^2 p$
    \end{itemize}
  \end{corollary}
  \begin{proof}
    То, что $J \vdash GL(<>\cdot \vee <>^2\cdot)$ является следствием
    следующих пунктов
    \begin{itemize}
      \item Используя предыдущее предложение и
            нормальность
            \begin{align*}
              J \vdash \Diamond ^2 p \rightarrow \Diamond ^2(p \wedge \neg \Diamond p \wedge \neg
              \Diamond ^2 p) \vee \Diamond ^2(p \wedge \Diamond p \wedge \neg \Diamond ^2 p)
            \end{align*}
            Так как $J \vdash \Diamond ^3 p \rightarrow \Diamond ^2 p$ и $K(\Diamond ) \vdash
            \Diamond p \wedge \neg \Diamond ^2 p \wedge \neg \Diamond ^3 p \rightarrow \Diamond (p
            \wedge \neg \Diamond p \wedge \neg \Diamond ^2 p)$
            \begin{align*}
              J \vdash \Diamond ^2(p \wedge \Diamond p \wedge \neg \Diamond ^2 p) \rightarrow
              \Diamond ^2\Diamond (p \wedge \neg \Diamond p \wedge \neg \Diamond ^2 p), \rightarrow
              \Diamond ^2(p \wedge \neg \Diamond p \wedge \neg \Diamond ^2 p)
            \end{align*}
      \item Заметим, что
            \begin{align*}
              K(\Diamond ) \vdash \Diamond p \rightarrow \Diamond (p \wedge \neg \Diamond p \wedge
              \neg \Diamond ^2 p) \vee \Diamond ^2 p \vee \Diamond ^3 p
            \end{align*}
            Откуда пользуясь тем, что $J \vdash \Diamond ^3 p \rightarrow
            \Diamond ^2 p$ получаем требуемое
    \end{itemize}
  \end{proof}
\end{document}
